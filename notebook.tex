
% Default to the notebook output style

    


% Inherit from the specified cell style.




    
\documentclass[11pt]{article}

    
    
    \usepackage[T1]{fontenc}
    % Nicer default font (+ math font) than Computer Modern for most use cases
    \usepackage{mathpazo}

    % Basic figure setup, for now with no caption control since it's done
    % automatically by Pandoc (which extracts ![](path) syntax from Markdown).
    \usepackage{graphicx}
    % We will generate all images so they have a width \maxwidth. This means
    % that they will get their normal width if they fit onto the page, but
    % are scaled down if they would overflow the margins.
    \makeatletter
    \def\maxwidth{\ifdim\Gin@nat@width>\linewidth\linewidth
    \else\Gin@nat@width\fi}
    \makeatother
    \let\Oldincludegraphics\includegraphics
    % Set max figure width to be 80% of text width, for now hardcoded.
    \renewcommand{\includegraphics}[1]{\Oldincludegraphics[width=.8\maxwidth]{#1}}
    % Ensure that by default, figures have no caption (until we provide a
    % proper Figure object with a Caption API and a way to capture that
    % in the conversion process - todo).
    \usepackage{caption}
    \DeclareCaptionLabelFormat{nolabel}{}
    \captionsetup{labelformat=nolabel}

    \usepackage{adjustbox} % Used to constrain images to a maximum size 
    \usepackage{xcolor} % Allow colors to be defined
    \usepackage{enumerate} % Needed for markdown enumerations to work
    \usepackage{geometry} % Used to adjust the document margins
    \usepackage{amsmath} % Equations
    \usepackage{amssymb} % Equations
    \usepackage{textcomp} % defines textquotesingle
    % Hack from http://tex.stackexchange.com/a/47451/13684:
    \AtBeginDocument{%
        \def\PYZsq{\textquotesingle}% Upright quotes in Pygmentized code
    }
    \usepackage{upquote} % Upright quotes for verbatim code
    \usepackage{eurosym} % defines \euro
    \usepackage[mathletters]{ucs} % Extended unicode (utf-8) support
    \usepackage[utf8x]{inputenc} % Allow utf-8 characters in the tex document
    \usepackage{fancyvrb} % verbatim replacement that allows latex
    \usepackage{grffile} % extends the file name processing of package graphics 
                         % to support a larger range 
    % The hyperref package gives us a pdf with properly built
    % internal navigation ('pdf bookmarks' for the table of contents,
    % internal cross-reference links, web links for URLs, etc.)
    \usepackage{hyperref}
    \usepackage{longtable} % longtable support required by pandoc >1.10
    \usepackage{booktabs}  % table support for pandoc > 1.12.2
    \usepackage[inline]{enumitem} % IRkernel/repr support (it uses the enumerate* environment)
    \usepackage[normalem]{ulem} % ulem is needed to support strikethroughs (\sout)
                                % normalem makes italics be italics, not underlines
    

    
    
    % Colors for the hyperref package
    \definecolor{urlcolor}{rgb}{0,.145,.698}
    \definecolor{linkcolor}{rgb}{.71,0.21,0.01}
    \definecolor{citecolor}{rgb}{.12,.54,.11}

    % ANSI colors
    \definecolor{ansi-black}{HTML}{3E424D}
    \definecolor{ansi-black-intense}{HTML}{282C36}
    \definecolor{ansi-red}{HTML}{E75C58}
    \definecolor{ansi-red-intense}{HTML}{B22B31}
    \definecolor{ansi-green}{HTML}{00A250}
    \definecolor{ansi-green-intense}{HTML}{007427}
    \definecolor{ansi-yellow}{HTML}{DDB62B}
    \definecolor{ansi-yellow-intense}{HTML}{B27D12}
    \definecolor{ansi-blue}{HTML}{208FFB}
    \definecolor{ansi-blue-intense}{HTML}{0065CA}
    \definecolor{ansi-magenta}{HTML}{D160C4}
    \definecolor{ansi-magenta-intense}{HTML}{A03196}
    \definecolor{ansi-cyan}{HTML}{60C6C8}
    \definecolor{ansi-cyan-intense}{HTML}{258F8F}
    \definecolor{ansi-white}{HTML}{C5C1B4}
    \definecolor{ansi-white-intense}{HTML}{A1A6B2}

    % commands and environments needed by pandoc snippets
    % extracted from the output of `pandoc -s`
    \providecommand{\tightlist}{%
      \setlength{\itemsep}{0pt}\setlength{\parskip}{0pt}}
    \DefineVerbatimEnvironment{Highlighting}{Verbatim}{commandchars=\\\{\}}
    % Add ',fontsize=\small' for more characters per line
    \newenvironment{Shaded}{}{}
    \newcommand{\KeywordTok}[1]{\textcolor[rgb]{0.00,0.44,0.13}{\textbf{{#1}}}}
    \newcommand{\DataTypeTok}[1]{\textcolor[rgb]{0.56,0.13,0.00}{{#1}}}
    \newcommand{\DecValTok}[1]{\textcolor[rgb]{0.25,0.63,0.44}{{#1}}}
    \newcommand{\BaseNTok}[1]{\textcolor[rgb]{0.25,0.63,0.44}{{#1}}}
    \newcommand{\FloatTok}[1]{\textcolor[rgb]{0.25,0.63,0.44}{{#1}}}
    \newcommand{\CharTok}[1]{\textcolor[rgb]{0.25,0.44,0.63}{{#1}}}
    \newcommand{\StringTok}[1]{\textcolor[rgb]{0.25,0.44,0.63}{{#1}}}
    \newcommand{\CommentTok}[1]{\textcolor[rgb]{0.38,0.63,0.69}{\textit{{#1}}}}
    \newcommand{\OtherTok}[1]{\textcolor[rgb]{0.00,0.44,0.13}{{#1}}}
    \newcommand{\AlertTok}[1]{\textcolor[rgb]{1.00,0.00,0.00}{\textbf{{#1}}}}
    \newcommand{\FunctionTok}[1]{\textcolor[rgb]{0.02,0.16,0.49}{{#1}}}
    \newcommand{\RegionMarkerTok}[1]{{#1}}
    \newcommand{\ErrorTok}[1]{\textcolor[rgb]{1.00,0.00,0.00}{\textbf{{#1}}}}
    \newcommand{\NormalTok}[1]{{#1}}
    
    % Additional commands for more recent versions of Pandoc
    \newcommand{\ConstantTok}[1]{\textcolor[rgb]{0.53,0.00,0.00}{{#1}}}
    \newcommand{\SpecialCharTok}[1]{\textcolor[rgb]{0.25,0.44,0.63}{{#1}}}
    \newcommand{\VerbatimStringTok}[1]{\textcolor[rgb]{0.25,0.44,0.63}{{#1}}}
    \newcommand{\SpecialStringTok}[1]{\textcolor[rgb]{0.73,0.40,0.53}{{#1}}}
    \newcommand{\ImportTok}[1]{{#1}}
    \newcommand{\DocumentationTok}[1]{\textcolor[rgb]{0.73,0.13,0.13}{\textit{{#1}}}}
    \newcommand{\AnnotationTok}[1]{\textcolor[rgb]{0.38,0.63,0.69}{\textbf{\textit{{#1}}}}}
    \newcommand{\CommentVarTok}[1]{\textcolor[rgb]{0.38,0.63,0.69}{\textbf{\textit{{#1}}}}}
    \newcommand{\VariableTok}[1]{\textcolor[rgb]{0.10,0.09,0.49}{{#1}}}
    \newcommand{\ControlFlowTok}[1]{\textcolor[rgb]{0.00,0.44,0.13}{\textbf{{#1}}}}
    \newcommand{\OperatorTok}[1]{\textcolor[rgb]{0.40,0.40,0.40}{{#1}}}
    \newcommand{\BuiltInTok}[1]{{#1}}
    \newcommand{\ExtensionTok}[1]{{#1}}
    \newcommand{\PreprocessorTok}[1]{\textcolor[rgb]{0.74,0.48,0.00}{{#1}}}
    \newcommand{\AttributeTok}[1]{\textcolor[rgb]{0.49,0.56,0.16}{{#1}}}
    \newcommand{\InformationTok}[1]{\textcolor[rgb]{0.38,0.63,0.69}{\textbf{\textit{{#1}}}}}
    \newcommand{\WarningTok}[1]{\textcolor[rgb]{0.38,0.63,0.69}{\textbf{\textit{{#1}}}}}
    
    
    % Define a nice break command that doesn't care if a line doesn't already
    % exist.
    \def\br{\hspace*{\fill} \\* }
    % Math Jax compatability definitions
    \def\gt{>}
    \def\lt{<}
    % Document parameters
    \title{Performance of Networked Systems 2}
    
    
    

    % Pygments definitions
    
\makeatletter
\def\PY@reset{\let\PY@it=\relax \let\PY@bf=\relax%
    \let\PY@ul=\relax \let\PY@tc=\relax%
    \let\PY@bc=\relax \let\PY@ff=\relax}
\def\PY@tok#1{\csname PY@tok@#1\endcsname}
\def\PY@toks#1+{\ifx\relax#1\empty\else%
    \PY@tok{#1}\expandafter\PY@toks\fi}
\def\PY@do#1{\PY@bc{\PY@tc{\PY@ul{%
    \PY@it{\PY@bf{\PY@ff{#1}}}}}}}
\def\PY#1#2{\PY@reset\PY@toks#1+\relax+\PY@do{#2}}

\expandafter\def\csname PY@tok@w\endcsname{\def\PY@tc##1{\textcolor[rgb]{0.73,0.73,0.73}{##1}}}
\expandafter\def\csname PY@tok@c\endcsname{\let\PY@it=\textit\def\PY@tc##1{\textcolor[rgb]{0.25,0.50,0.50}{##1}}}
\expandafter\def\csname PY@tok@cp\endcsname{\def\PY@tc##1{\textcolor[rgb]{0.74,0.48,0.00}{##1}}}
\expandafter\def\csname PY@tok@k\endcsname{\let\PY@bf=\textbf\def\PY@tc##1{\textcolor[rgb]{0.00,0.50,0.00}{##1}}}
\expandafter\def\csname PY@tok@kp\endcsname{\def\PY@tc##1{\textcolor[rgb]{0.00,0.50,0.00}{##1}}}
\expandafter\def\csname PY@tok@kt\endcsname{\def\PY@tc##1{\textcolor[rgb]{0.69,0.00,0.25}{##1}}}
\expandafter\def\csname PY@tok@o\endcsname{\def\PY@tc##1{\textcolor[rgb]{0.40,0.40,0.40}{##1}}}
\expandafter\def\csname PY@tok@ow\endcsname{\let\PY@bf=\textbf\def\PY@tc##1{\textcolor[rgb]{0.67,0.13,1.00}{##1}}}
\expandafter\def\csname PY@tok@nb\endcsname{\def\PY@tc##1{\textcolor[rgb]{0.00,0.50,0.00}{##1}}}
\expandafter\def\csname PY@tok@nf\endcsname{\def\PY@tc##1{\textcolor[rgb]{0.00,0.00,1.00}{##1}}}
\expandafter\def\csname PY@tok@nc\endcsname{\let\PY@bf=\textbf\def\PY@tc##1{\textcolor[rgb]{0.00,0.00,1.00}{##1}}}
\expandafter\def\csname PY@tok@nn\endcsname{\let\PY@bf=\textbf\def\PY@tc##1{\textcolor[rgb]{0.00,0.00,1.00}{##1}}}
\expandafter\def\csname PY@tok@ne\endcsname{\let\PY@bf=\textbf\def\PY@tc##1{\textcolor[rgb]{0.82,0.25,0.23}{##1}}}
\expandafter\def\csname PY@tok@nv\endcsname{\def\PY@tc##1{\textcolor[rgb]{0.10,0.09,0.49}{##1}}}
\expandafter\def\csname PY@tok@no\endcsname{\def\PY@tc##1{\textcolor[rgb]{0.53,0.00,0.00}{##1}}}
\expandafter\def\csname PY@tok@nl\endcsname{\def\PY@tc##1{\textcolor[rgb]{0.63,0.63,0.00}{##1}}}
\expandafter\def\csname PY@tok@ni\endcsname{\let\PY@bf=\textbf\def\PY@tc##1{\textcolor[rgb]{0.60,0.60,0.60}{##1}}}
\expandafter\def\csname PY@tok@na\endcsname{\def\PY@tc##1{\textcolor[rgb]{0.49,0.56,0.16}{##1}}}
\expandafter\def\csname PY@tok@nt\endcsname{\let\PY@bf=\textbf\def\PY@tc##1{\textcolor[rgb]{0.00,0.50,0.00}{##1}}}
\expandafter\def\csname PY@tok@nd\endcsname{\def\PY@tc##1{\textcolor[rgb]{0.67,0.13,1.00}{##1}}}
\expandafter\def\csname PY@tok@s\endcsname{\def\PY@tc##1{\textcolor[rgb]{0.73,0.13,0.13}{##1}}}
\expandafter\def\csname PY@tok@sd\endcsname{\let\PY@it=\textit\def\PY@tc##1{\textcolor[rgb]{0.73,0.13,0.13}{##1}}}
\expandafter\def\csname PY@tok@si\endcsname{\let\PY@bf=\textbf\def\PY@tc##1{\textcolor[rgb]{0.73,0.40,0.53}{##1}}}
\expandafter\def\csname PY@tok@se\endcsname{\let\PY@bf=\textbf\def\PY@tc##1{\textcolor[rgb]{0.73,0.40,0.13}{##1}}}
\expandafter\def\csname PY@tok@sr\endcsname{\def\PY@tc##1{\textcolor[rgb]{0.73,0.40,0.53}{##1}}}
\expandafter\def\csname PY@tok@ss\endcsname{\def\PY@tc##1{\textcolor[rgb]{0.10,0.09,0.49}{##1}}}
\expandafter\def\csname PY@tok@sx\endcsname{\def\PY@tc##1{\textcolor[rgb]{0.00,0.50,0.00}{##1}}}
\expandafter\def\csname PY@tok@m\endcsname{\def\PY@tc##1{\textcolor[rgb]{0.40,0.40,0.40}{##1}}}
\expandafter\def\csname PY@tok@gh\endcsname{\let\PY@bf=\textbf\def\PY@tc##1{\textcolor[rgb]{0.00,0.00,0.50}{##1}}}
\expandafter\def\csname PY@tok@gu\endcsname{\let\PY@bf=\textbf\def\PY@tc##1{\textcolor[rgb]{0.50,0.00,0.50}{##1}}}
\expandafter\def\csname PY@tok@gd\endcsname{\def\PY@tc##1{\textcolor[rgb]{0.63,0.00,0.00}{##1}}}
\expandafter\def\csname PY@tok@gi\endcsname{\def\PY@tc##1{\textcolor[rgb]{0.00,0.63,0.00}{##1}}}
\expandafter\def\csname PY@tok@gr\endcsname{\def\PY@tc##1{\textcolor[rgb]{1.00,0.00,0.00}{##1}}}
\expandafter\def\csname PY@tok@ge\endcsname{\let\PY@it=\textit}
\expandafter\def\csname PY@tok@gs\endcsname{\let\PY@bf=\textbf}
\expandafter\def\csname PY@tok@gp\endcsname{\let\PY@bf=\textbf\def\PY@tc##1{\textcolor[rgb]{0.00,0.00,0.50}{##1}}}
\expandafter\def\csname PY@tok@go\endcsname{\def\PY@tc##1{\textcolor[rgb]{0.53,0.53,0.53}{##1}}}
\expandafter\def\csname PY@tok@gt\endcsname{\def\PY@tc##1{\textcolor[rgb]{0.00,0.27,0.87}{##1}}}
\expandafter\def\csname PY@tok@err\endcsname{\def\PY@bc##1{\setlength{\fboxsep}{0pt}\fcolorbox[rgb]{1.00,0.00,0.00}{1,1,1}{\strut ##1}}}
\expandafter\def\csname PY@tok@kc\endcsname{\let\PY@bf=\textbf\def\PY@tc##1{\textcolor[rgb]{0.00,0.50,0.00}{##1}}}
\expandafter\def\csname PY@tok@kd\endcsname{\let\PY@bf=\textbf\def\PY@tc##1{\textcolor[rgb]{0.00,0.50,0.00}{##1}}}
\expandafter\def\csname PY@tok@kn\endcsname{\let\PY@bf=\textbf\def\PY@tc##1{\textcolor[rgb]{0.00,0.50,0.00}{##1}}}
\expandafter\def\csname PY@tok@kr\endcsname{\let\PY@bf=\textbf\def\PY@tc##1{\textcolor[rgb]{0.00,0.50,0.00}{##1}}}
\expandafter\def\csname PY@tok@bp\endcsname{\def\PY@tc##1{\textcolor[rgb]{0.00,0.50,0.00}{##1}}}
\expandafter\def\csname PY@tok@fm\endcsname{\def\PY@tc##1{\textcolor[rgb]{0.00,0.00,1.00}{##1}}}
\expandafter\def\csname PY@tok@vc\endcsname{\def\PY@tc##1{\textcolor[rgb]{0.10,0.09,0.49}{##1}}}
\expandafter\def\csname PY@tok@vg\endcsname{\def\PY@tc##1{\textcolor[rgb]{0.10,0.09,0.49}{##1}}}
\expandafter\def\csname PY@tok@vi\endcsname{\def\PY@tc##1{\textcolor[rgb]{0.10,0.09,0.49}{##1}}}
\expandafter\def\csname PY@tok@vm\endcsname{\def\PY@tc##1{\textcolor[rgb]{0.10,0.09,0.49}{##1}}}
\expandafter\def\csname PY@tok@sa\endcsname{\def\PY@tc##1{\textcolor[rgb]{0.73,0.13,0.13}{##1}}}
\expandafter\def\csname PY@tok@sb\endcsname{\def\PY@tc##1{\textcolor[rgb]{0.73,0.13,0.13}{##1}}}
\expandafter\def\csname PY@tok@sc\endcsname{\def\PY@tc##1{\textcolor[rgb]{0.73,0.13,0.13}{##1}}}
\expandafter\def\csname PY@tok@dl\endcsname{\def\PY@tc##1{\textcolor[rgb]{0.73,0.13,0.13}{##1}}}
\expandafter\def\csname PY@tok@s2\endcsname{\def\PY@tc##1{\textcolor[rgb]{0.73,0.13,0.13}{##1}}}
\expandafter\def\csname PY@tok@sh\endcsname{\def\PY@tc##1{\textcolor[rgb]{0.73,0.13,0.13}{##1}}}
\expandafter\def\csname PY@tok@s1\endcsname{\def\PY@tc##1{\textcolor[rgb]{0.73,0.13,0.13}{##1}}}
\expandafter\def\csname PY@tok@mb\endcsname{\def\PY@tc##1{\textcolor[rgb]{0.40,0.40,0.40}{##1}}}
\expandafter\def\csname PY@tok@mf\endcsname{\def\PY@tc##1{\textcolor[rgb]{0.40,0.40,0.40}{##1}}}
\expandafter\def\csname PY@tok@mh\endcsname{\def\PY@tc##1{\textcolor[rgb]{0.40,0.40,0.40}{##1}}}
\expandafter\def\csname PY@tok@mi\endcsname{\def\PY@tc##1{\textcolor[rgb]{0.40,0.40,0.40}{##1}}}
\expandafter\def\csname PY@tok@il\endcsname{\def\PY@tc##1{\textcolor[rgb]{0.40,0.40,0.40}{##1}}}
\expandafter\def\csname PY@tok@mo\endcsname{\def\PY@tc##1{\textcolor[rgb]{0.40,0.40,0.40}{##1}}}
\expandafter\def\csname PY@tok@ch\endcsname{\let\PY@it=\textit\def\PY@tc##1{\textcolor[rgb]{0.25,0.50,0.50}{##1}}}
\expandafter\def\csname PY@tok@cm\endcsname{\let\PY@it=\textit\def\PY@tc##1{\textcolor[rgb]{0.25,0.50,0.50}{##1}}}
\expandafter\def\csname PY@tok@cpf\endcsname{\let\PY@it=\textit\def\PY@tc##1{\textcolor[rgb]{0.25,0.50,0.50}{##1}}}
\expandafter\def\csname PY@tok@c1\endcsname{\let\PY@it=\textit\def\PY@tc##1{\textcolor[rgb]{0.25,0.50,0.50}{##1}}}
\expandafter\def\csname PY@tok@cs\endcsname{\let\PY@it=\textit\def\PY@tc##1{\textcolor[rgb]{0.25,0.50,0.50}{##1}}}

\def\PYZbs{\char`\\}
\def\PYZus{\char`\_}
\def\PYZob{\char`\{}
\def\PYZcb{\char`\}}
\def\PYZca{\char`\^}
\def\PYZam{\char`\&}
\def\PYZlt{\char`\<}
\def\PYZgt{\char`\>}
\def\PYZsh{\char`\#}
\def\PYZpc{\char`\%}
\def\PYZdl{\char`\$}
\def\PYZhy{\char`\-}
\def\PYZsq{\char`\'}
\def\PYZdq{\char`\"}
\def\PYZti{\char`\~}
% for compatibility with earlier versions
\def\PYZat{@}
\def\PYZlb{[}
\def\PYZrb{]}
\makeatother


    % Exact colors from NB
    \definecolor{incolor}{rgb}{0.0, 0.0, 0.5}
    \definecolor{outcolor}{rgb}{0.545, 0.0, 0.0}



    
    % Prevent overflowing lines due to hard-to-break entities
    \sloppy 
    % Setup hyperref package
    \hypersetup{
      breaklinks=true,  % so long urls are correctly broken across lines
      colorlinks=true,
      urlcolor=urlcolor,
      linkcolor=linkcolor,
      citecolor=citecolor,
      }
    % Slightly bigger margins than the latex defaults
    
    \geometry{verbose,tmargin=1in,bmargin=1in,lmargin=1in,rmargin=1in}
    
    

    \begin{document}
    
    
    \maketitle
    
    

    
    \section{Dimensioning of cellular networks with video-conferencing
services}\label{dimensioning-of-cellular-networks-with-video-conferencing-services}

A mobile operator of a cellular GSM network wants to determine how many
base stations are needed to satisfy its customers' Quality of Service
(QoS) demands. To this end, the operator wants to determine the maximum
size of a cell for which the call-blocking probability is still below
some given threshold. Voice telephone calls are generated with rate 10
calls per minute per square kilometer (i.e., km2 ), and the call
duration has an exponential distribution with mean 1.5 minutes. Assume
that each voice call requires a single channel to the nearest base
station, and that each cell can support only 6 channels in parallel.

To make a proper decision on the number of base stations to be placed to
offer good quality to its customers, the operator wants to understand
the impact of the cell size (in km2 ) on the call-blocking probability.

    \subsection{Formulate a simple model description for the
problem.}\label{formulate-a-simple-model-description-for-the-problem.}

    When describing a model we can use Kendall's shorthand notation to
characterize queueing models. This notation consists of three different
aspects, and one extra:

\begin{enumerate}
\def\labelenumi{\arabic{enumi}.}
\tightlist
\item
  Interarrival time distribution
\item
  Service time distribution
\item
  Number of servers
\item
  Possible queueing capacity (which is 0 because there is blocking
  instead of queueing)
\end{enumerate}

For the first two aspects we can use three letters:

\begin{itemize}
\tightlist
\item
  G: General distribution
\item
  M: Exponential distribution (also called memoyless or Makov)
\item
  D: Deterministic
\end{itemize}

Acording to this discription of Kendall's shorthand notation we can
describe the model in this assignment to be of the M/M/6/0 variety,
because both the interarrival times and the service time have a
memoryless property.

    \subsection{Give a formula for the call blocking probability in terms of
the model
parameters.}\label{give-a-formula-for-the-call-blocking-probability-in-terms-of-the-model-parameters.}

    \subsubsection{The "Erlang-B formula"}\label{the-erlang-b-formula}

The Erlang-B formula is used for calculating the blocking probability
that describes the probability of losing calls for a group of identical
parallel resources (also known as M/M/c/c queue). The formula is given
below:

\[ \frac{\frac{(\lambda\beta)^N}{N!}}{1 + \frac{(\lambda\beta)^1}{1!} + \frac{(\lambda\beta)^2}{2!} + ... + \frac{(\lambda\beta)^N}{N!}} \]

The formula does make a few assumptions:

\begin{itemize}
\tightlist
\item
  It assumes a Poisson process for the arival instances, so that they
  are independent.
\item
  The holding (waiting) times are exponentially distributed.
\item
  The formula assumes an infinite population of sources (e.g.
  phonecalls).
\item
  The arrival rate (lambda) is constant.
\item
  The arrival rate is independent of the number of active sources.
\item
  Number of resources are assumed to be infinite.
\item
  No denied request will be kept (buffer-less loss system). So there are
  no queues.
\end{itemize}

\subsubsection{Parameters}\label{parameters}

\paragraph{Lamda (λ) is the rate per minute per square
kilometer}\label{lamda-ux3bb-is-the-rate-per-minute-per-square-kilometer}

lamda = 10

\paragraph{Beta (β) is the exponential distribution with mean 1.5
minutes}\label{beta-ux3b2-is-the-exponential-distribution-with-mean-1.5-minutes}

beta = 1.5

\paragraph{Amount of channels in parallel
(N)}\label{amount-of-channels-in-parallel-n}

N = 6

    \subsection{Write your own `Erlang-B calculator'' and calculate the
blocking probability for cell sizes 0.5 km2, 1km2 and 2
km2.}\label{write-your-own-erlang-b-calculator-and-calculate-the-blocking-probability-for-cell-sizes-0.5-km2-1km2-and-2-km2.}

    \begin{Verbatim}[commandchars=\\\{\}]
{\color{incolor}In [{\color{incolor}21}]:} \PY{k+kn}{from} \PY{n+nn}{ipywidgets} \PY{k}{import} \PY{n}{interact}
         \PY{k+kn}{import} \PY{n+nn}{numpy} \PY{k}{as} \PY{n+nn}{np}
         \PY{k+kn}{import} \PY{n+nn}{math}
         
         \PY{k+kn}{from} \PY{n+nn}{bokeh}\PY{n+nn}{.}\PY{n+nn}{io} \PY{k}{import} \PY{n}{push\PYZus{}notebook}\PY{p}{,} \PY{n}{show}\PY{p}{,} \PY{n}{output\PYZus{}notebook}
         \PY{k+kn}{from} \PY{n+nn}{bokeh}\PY{n+nn}{.}\PY{n+nn}{plotting} \PY{k}{import} \PY{n}{figure}
         \PY{k+kn}{from} \PY{n+nn}{bokeh}\PY{n+nn}{.}\PY{n+nn}{models} \PY{k}{import} \PY{n}{ColumnDataSource}\PY{p}{,} \PY{n}{HoverTool}\PY{p}{,} \PY{n}{LassoSelectTool}\PY{p}{,} \PY{n}{WheelZoomTool}
         \PY{k+kn}{from} \PY{n+nn}{bokeh}\PY{n+nn}{.}\PY{n+nn}{palettes} \PY{k}{import} \PY{n}{Spectral3}
         
         \PY{n}{output\PYZus{}notebook}\PY{p}{(}\PY{p}{)}
\end{Verbatim}


    
    
    
    
    \begin{Verbatim}[commandchars=\\\{\}]
{\color{incolor}In [{\color{incolor}22}]:} \PY{c+c1}{\PYZsh{} Calculate upper part of Erlang\PYZhy{}B formula}
         \PY{k}{def} \PY{n+nf}{bp\PYZus{}calc}\PY{p}{(}\PY{n}{lamda}\PY{p}{,} \PY{n}{beta}\PY{p}{,} \PY{n}{N}\PY{p}{)}\PY{p}{:}
             \PY{n}{bp} \PY{o}{=} \PY{p}{(}\PY{n}{lamda} \PY{o}{*} \PY{n}{beta}\PY{p}{)} \PY{o}{*}\PY{o}{*} \PY{n}{N} \PY{o}{/} \PY{n}{math}\PY{o}{.}\PY{n}{factorial}\PY{p}{(}\PY{n}{N}\PY{p}{)}
             \PY{k}{return} \PY{n}{bp}
         
         \PY{c+c1}{\PYZsh{} Calculate Markov chain}
         \PY{k}{def} \PY{n+nf}{erlang\PYZus{}B}\PY{p}{(}\PY{n}{lamda}\PY{p}{,} \PY{n}{beta}\PY{p}{,} \PY{n}{N}\PY{p}{)}\PY{p}{:}
             \PY{n}{bp1} \PY{o}{=} \PY{n}{bp\PYZus{}calc}\PY{p}{(}\PY{n}{lamda}\PY{p}{,} \PY{n}{beta}\PY{p}{,} \PY{n}{N}\PY{p}{)}
             \PY{n}{bp2} \PY{o}{=} \PY{l+m+mi}{1}
             
             \PY{k}{for} \PY{n}{i} \PY{o+ow}{in} \PY{n+nb}{range}\PY{p}{(}\PY{n}{N}\PY{p}{)}\PY{p}{:}
                 \PY{n}{bp2} \PY{o}{+}\PY{o}{=} \PY{n}{bp\PYZus{}calc}\PY{p}{(}\PY{n}{lamda}\PY{p}{,} \PY{n}{beta}\PY{p}{,} \PY{n}{i} \PY{o}{+} \PY{l+m+mi}{1}\PY{p}{)}
                 
             \PY{k}{return} \PY{n}{bp1} \PY{o}{/} \PY{n}{bp2}
\end{Verbatim}


    \begin{Verbatim}[commandchars=\\\{\}]
{\color{incolor}In [{\color{incolor}23}]:} \PY{c+c1}{\PYZsh{} Initializa data}
         \PY{n}{data\PYZus{}x} \PY{o}{=} \PY{p}{[}\PY{l+s+s1}{\PYZsq{}}\PY{l+s+s1}{0.5 km2}\PY{l+s+s1}{\PYZsq{}}\PY{p}{,}\PY{l+s+s1}{\PYZsq{}}\PY{l+s+s1}{1 km2}\PY{l+s+s1}{\PYZsq{}}\PY{p}{,}\PY{l+s+s1}{\PYZsq{}}\PY{l+s+s1}{2 km2}\PY{l+s+s1}{\PYZsq{}}\PY{p}{]}
         \PY{n}{data\PYZus{}y} \PY{o}{=} \PY{p}{[}\PY{p}{]}
         
         \PY{c+c1}{\PYZsh{} Initialize variable}
         \PY{n}{datapoints} \PY{o}{=} \PY{p}{[}\PY{l+m+mf}{0.5}\PY{p}{,}\PY{l+m+mi}{1}\PY{p}{,}\PY{l+m+mi}{2}\PY{p}{]}
         \PY{n}{lamda} \PY{o}{=} \PY{l+m+mi}{10}
         \PY{n}{beta} \PY{o}{=} \PY{l+m+mf}{1.5}
         
         \PY{c+c1}{\PYZsh{} Fill data}
         \PY{k}{for} \PY{n}{i} \PY{o+ow}{in} \PY{n}{datapoints}\PY{p}{:}
             \PY{n}{data\PYZus{}y}\PY{o}{.}\PY{n}{append}\PY{p}{(}\PY{n}{erlang\PYZus{}B}\PY{p}{(}\PY{n}{lamda} \PY{o}{*} \PY{n}{i}\PY{p}{,} \PY{n}{beta}\PY{p}{,} \PY{l+m+mi}{6}\PY{p}{)}\PY{p}{)}
\end{Verbatim}


    \begin{Verbatim}[commandchars=\\\{\}]
{\color{incolor}In [{\color{incolor}24}]:} \PY{c+c1}{\PYZsh{} Setting data in right format}
         \PY{n}{source} \PY{o}{=} \PY{n}{ColumnDataSource}\PY{p}{(}\PY{n}{data}\PY{o}{=}\PY{n+nb}{dict}\PY{p}{(}\PY{n}{x\PYZus{}axis}\PY{o}{=}\PY{n}{data\PYZus{}x}\PY{p}{,} 
                                             \PY{n}{y\PYZus{}axis}\PY{o}{=}\PY{n}{data\PYZus{}y}\PY{p}{,} 
                                             \PY{n}{color}\PY{o}{=}\PY{n}{Spectral3}\PY{p}{)}\PY{p}{)}
         
         \PY{c+c1}{\PYZsh{} Creating tooltip}
         \PY{n}{hover} \PY{o}{=} \PY{n}{HoverTool}\PY{p}{(}\PY{n}{tooltips}\PY{o}{=}\PY{p}{[}
             \PY{p}{(}\PY{l+s+s2}{\PYZdq{}}\PY{l+s+s2}{Cell Size}\PY{l+s+s2}{\PYZdq{}}\PY{p}{,} \PY{l+s+s2}{\PYZdq{}}\PY{l+s+s2}{@x\PYZus{}axis}\PY{l+s+s2}{\PYZdq{}}\PY{p}{)}\PY{p}{,}
             \PY{p}{(}\PY{l+s+s2}{\PYZdq{}}\PY{l+s+s2}{Blocking Probability}\PY{l+s+s2}{\PYZdq{}}\PY{p}{,} \PY{l+s+s2}{\PYZdq{}}\PY{l+s+s2}{@y\PYZus{}axis}\PY{l+s+s2}{\PYZdq{}}\PY{p}{)}\PY{p}{,}
         \PY{p}{]}\PY{p}{)}
         
         \PY{c+c1}{\PYZsh{} Create figure}
         \PY{n}{p} \PY{o}{=} \PY{n}{figure}\PY{p}{(}\PY{n}{x\PYZus{}range}\PY{o}{=}\PY{n}{data\PYZus{}x}\PY{p}{,} \PY{n}{y\PYZus{}range}\PY{o}{=}\PY{p}{(}\PY{l+m+mi}{0}\PY{p}{,}\PY{p}{(}\PY{n+nb}{max}\PY{p}{(}\PY{n}{data\PYZus{}y}\PY{p}{)} \PY{o}{+} \PY{p}{(}\PY{n+nb}{max}\PY{p}{(}\PY{n}{data\PYZus{}y}\PY{p}{)} \PY{o}{/} \PY{l+m+mi}{10}\PY{p}{)}\PY{p}{)}\PY{p}{)}\PY{p}{,} 
                    \PY{n}{plot\PYZus{}height}\PY{o}{=}\PY{l+m+mi}{350}\PY{p}{,} \PY{n}{title}\PY{o}{=}\PY{l+s+s2}{\PYZdq{}}\PY{l+s+s2}{Erlang Blocking Calculator}\PY{l+s+s2}{\PYZdq{}}\PY{p}{,}
                    \PY{n}{tools}\PY{o}{=}\PY{p}{[}\PY{n}{hover}\PY{p}{,}\PY{n}{LassoSelectTool}\PY{p}{(}\PY{p}{)}\PY{p}{,} \PY{n}{WheelZoomTool}\PY{p}{(}\PY{p}{)}\PY{p}{]}\PY{p}{)}
         
         \PY{c+c1}{\PYZsh{} Create bars}
         \PY{n}{p}\PY{o}{.}\PY{n}{vbar}\PY{p}{(}\PY{n}{x}\PY{o}{=}\PY{l+s+s1}{\PYZsq{}}\PY{l+s+s1}{x\PYZus{}axis}\PY{l+s+s1}{\PYZsq{}}\PY{p}{,} \PY{n}{top}\PY{o}{=}\PY{l+s+s1}{\PYZsq{}}\PY{l+s+s1}{y\PYZus{}axis}\PY{l+s+s1}{\PYZsq{}}\PY{p}{,} \PY{n}{width}\PY{o}{=}\PY{l+m+mf}{0.9}\PY{p}{,} \PY{n}{color}\PY{o}{=}\PY{l+s+s1}{\PYZsq{}}\PY{l+s+s1}{color}\PY{l+s+s1}{\PYZsq{}}\PY{p}{,} 
                \PY{n}{legend}\PY{o}{=}\PY{l+s+s2}{\PYZdq{}}\PY{l+s+s2}{x\PYZus{}axis}\PY{l+s+s2}{\PYZdq{}}\PY{p}{,} \PY{n}{source}\PY{o}{=}\PY{n}{source}\PY{p}{)}
         
         \PY{c+c1}{\PYZsh{} Colors and legend}
         \PY{n}{p}\PY{o}{.}\PY{n}{xgrid}\PY{o}{.}\PY{n}{grid\PYZus{}line\PYZus{}color} \PY{o}{=} \PY{k+kc}{None}
         \PY{n}{p}\PY{o}{.}\PY{n}{legend}\PY{o}{.}\PY{n}{orientation} \PY{o}{=} \PY{l+s+s2}{\PYZdq{}}\PY{l+s+s2}{horizontal}\PY{l+s+s2}{\PYZdq{}}
         \PY{n}{p}\PY{o}{.}\PY{n}{legend}\PY{o}{.}\PY{n}{location} \PY{o}{=} \PY{l+s+s2}{\PYZdq{}}\PY{l+s+s2}{top\PYZus{}center}\PY{l+s+s2}{\PYZdq{}}
\end{Verbatim}


    \begin{Verbatim}[commandchars=\\\{\}]
{\color{incolor}In [{\color{incolor}25}]:} \PY{c+c1}{\PYZsh{} Show graph}
         \PY{n}{show}\PY{p}{(}\PY{n}{p}\PY{p}{)}
\end{Verbatim}


    
    
    
    
    \subsection{The call blocking probability is known to be insensitive
with respect to the distribution of the call duration. What exactly does
that
mean?}\label{the-call-blocking-probability-is-known-to-be-insensitive-with-respect-to-the-distribution-of-the-call-duration.-what-exactly-does-that-mean}

    This means that the formula is also valid for non-exponential call
holding times.

    \subsection{Is the call blocking probability also insensitive with
respect to the inter-arrival time distribution of the calls? If so, why,
if not so, give a
counter-example.}\label{is-the-call-blocking-probability-also-insensitive-with-respect-to-the-inter-arrival-time-distribution-of-the-calls-if-so-why-if-not-so-give-a-counter-example.}

    The call blocking probability is not insensitive with respect to the
inter-arrival time distribution of the calls. This is because Erlang
Blocking Formula requires Poisson arrival times. These Poisson arrival
times have to satisfy two requirements.

\begin{enumerate}
\def\labelenumi{\arabic{enumi}.}
\tightlist
\item
  Mutually independent
\item
  Exponentially distributed
\end{enumerate}

So by this definition the inter-arrival time distribution has to be, by
definition, exponentially distributed.

    \subsubsection{Now suppose the service provider wants to offer a new
additional service to its customers: video conferencing. Each conference
call requires 3 parallel channels for each connection. Video
conferencing calls arrive according to a Poisson process with rate 0.25
calls per hour per km2, and the conference call duration is
exponentially distributed with mean 15 minutes. Recall that each cell
has 6 channels. Call attempts are blocked when there are not enough
lines available. Assume throughout that the cell size is 1
km2.}\label{now-suppose-the-service-provider-wants-to-offer-a-new-additional-service-to-its-customers-video-conferencing.-each-conference-call-requires-3-parallel-channels-for-each-connection.-video-conferencing-calls-arrive-according-to-a-poisson-process-with-rate-0.25-calls-per-hour-per-km2-and-the-conference-call-duration-is-exponentially-distributed-with-mean-15-minutes.-recall-that-each-cell-has-6-channels.-call-attempts-are-blocked-when-there-are-not-enough-lines-available.-assume-throughout-that-the-cell-size-is-1-km2.}

    \subsection{Formulate the evolution of the system as a two-dimensional
continuous-time Markov chain, describing the numbers of calls at both
call types (i.e., voice calls and video conferencing calls,
respectively).}\label{formulate-the-evolution-of-the-system-as-a-two-dimensional-continuous-time-markov-chain-describing-the-numbers-of-calls-at-both-call-types-i.e.-voice-calls-and-video-conferencing-calls-respectively.}

    The two dimensional continuous-time Markov chain looks like figure
(ref). This models follows these parameters:

\begin{itemize}
\tightlist
\item
  Number of types N: 2
\item
  Number of channels C: 6
\item
  b1, Cost of Type 1 (voice calls): 1
\item
  b2, Cost of Type 2 (video conferencing): 3
\item
  Lambda1: 10
\item
  Lambda2: 0.25/60 = 0.00416
\item
  mu1 = 1/d1 = 1/1.5 = 0.666
\item
  mu2 = 1/d2 = 1/15 = 0.0666
\end{itemize}

In figure (ref) the probability that a video conferencing call is added
to the system is depicted by lambda2 (INSERT PROPER CHAR) and the
probability of a voice call being added to the system is depicted by
lambda1 (INSERT). The probability that a call is finished is depicted by
mu1 of 2 depending on the type, times the amount of calls of the same
type.

    \subsection{Formulate and solve the balance equations for the Markov
chain and calculate the equilibrium state probability for each
state.}\label{formulate-and-solve-the-balance-equations-for-the-markov-chain-and-calculate-the-equilibrium-state-probability-for-each-state.}

    State space = \{(0,0), (0,1), (0,2), (1,1), (2,1), (3,1), (1,0), (2,0),
(3,0), (4,0), (5,0), (6,0)\} (12 states)

Balance equations:

\begin{enumerate}
\def\labelenumi{\arabic{enumi}.}
\tightlist
\item
  For state (0,0): \emph{(lambda1 + lambda2) * pi(0,0) = mu1 * pi(1,0) +
  mu2 * pi(0,1)}
\item
  For state (0,1): \emph{(lambda1 + mu2 + lambda2) * pi(0,1) = mu1 *
  pi(1,1) + (2 * mu2) * pi(0,2) + lambda2 * pi(0, 0)}
\item
  For state (0,2): \emph{(2 * mu2) * pi(0,2) = lambda2 * pi(0,1)}
\item
  For state (1,1): \emph{(lambda1 + mu2 + mu1) * pi(1,1) = (2 * mu1) *
  pi(2,1) + lambda1 * pi(0,1) + lambda2 * pi(1,0)}
\item
  For state (2,1): \emph{(lambda1 + mu2 + 2 * mu1) * pi(2,1) = (3 * mu1)
  * pi(3,1) + lambda1 * pi(1,1) + lambda2 * pi(2,0)}
\item
  For state (3,1): \emph{(mu2 + 3 * mu1) * pi(3,1) = lambda1 * pi(2,1) +
  lambda2 * pi(3.0)}
\item
  For state (1,0): \emph{(lambda1 + lambda2 + mu1) * pi(1,0) = lambda1 *
  p(0,0) + mu2 * p(1,1) + 2* mu1 * p(2,0)}
\item
  For state (2,0): \emph{(lambda1 + lambda2 + 2 * mu1) * p(2,0) =
  lambda1 * pi(1,0) + mu2 * pi(2,1) + 3 * mu1 * p(3,0)}
\item
  For state (3,0): \emph{(lambda1 + 3 * mu1 + lambda2) * pi(3,0) =
  lambda1 * pi(2,0) + (4 * mu1) * pi(4,0) + mu2 * pi(3,1)}
\item
  For state (4,0): \emph{(lambda1 + 4 * mu1) * pi(4,0) = lambda1 *
  pi(3,0) + (5 * mu1) * pi(5,0)}
\item
  For state (5,0): \emph{(lambda1 + 5 * mu1) * pi(5,0) = lambda1 *
  pi(4,0) + (6 * mu1) * pi(6,0)}
\item
  For state (6,0): \emph{(6 * mu1) * pi(6,0) = lambda1 * pi(5,0)}
\end{enumerate}

    \begin{Verbatim}[commandchars=\\\{\}]
{\color{incolor}In [{\color{incolor}26}]:} \PY{k+kn}{import} \PY{n+nn}{numpy}
         
         \PY{n}{lambda1} \PY{o}{=} \PY{l+m+mi}{10}
         \PY{n}{lambda2} \PY{o}{=}  \PY{l+m+mf}{0.25}\PY{o}{/}\PY{l+m+mi}{60}
         \PY{n}{mu1} \PY{o}{=}  \PY{l+m+mi}{1}\PY{o}{/}\PY{l+m+mf}{1.5}
         \PY{n}{mu2} \PY{o}{=} \PY{l+m+mi}{1}\PY{o}{/}\PY{l+m+mi}{15}
         
         \PY{c+c1}{\PYZsh{} State space = \PYZob{}(0,0), (0,1), (0,2), (1,1), (2,1), (3,1), (1,0), (2,0), (3,0), (4,0), (5,0), (6,0)\PYZcb{} (12 states) }
         \PY{c+c1}{\PYZsh{} [0, 0 ,0, 0 , 0 ,0 , 0 , 0 , 0 , 0, 0, 0 ]}
         \PY{n}{e00} \PY{o}{=} \PY{p}{[}\PY{n}{lambda1} \PY{o}{+} \PY{n}{lambda2}\PY{p}{,} \PY{n}{mu2} \PY{p}{,}\PY{l+m+mi}{0}\PY{p}{,} \PY{l+m+mi}{0} \PY{p}{,} \PY{l+m+mi}{0} \PY{p}{,} \PY{l+m+mi}{0} \PY{p}{,} \PY{o}{\PYZhy{}}\PY{n}{mu1} \PY{p}{,} \PY{l+m+mi}{0} \PY{p}{,} \PY{l+m+mi}{0} \PY{p}{,} \PY{l+m+mi}{0}\PY{p}{,} \PY{l+m+mi}{0}\PY{p}{,} \PY{l+m+mi}{0} \PY{p}{]}
         \PY{n}{e01} \PY{o}{=} \PY{p}{[}\PY{o}{\PYZhy{}}\PY{n}{lambda2}\PY{p}{,} \PY{n}{lambda1} \PY{o}{+} \PY{n}{mu2} \PY{o}{+} \PY{n}{lambda2} \PY{p}{,} \PY{o}{\PYZhy{}}\PY{l+m+mi}{2} \PY{o}{*} \PY{n}{mu2}\PY{p}{,} \PY{o}{\PYZhy{}}\PY{n}{mu1} \PY{p}{,} \PY{l+m+mi}{0} \PY{p}{,} \PY{l+m+mi}{0} \PY{p}{,} \PY{l+m+mi}{0} \PY{p}{,} \PY{l+m+mi}{0} \PY{p}{,} \PY{l+m+mi}{0} \PY{p}{,} \PY{l+m+mi}{0}\PY{p}{,} \PY{l+m+mi}{0}\PY{p}{,} \PY{l+m+mi}{0} \PY{p}{]}
         \PY{n}{e02} \PY{o}{=} \PY{p}{[}\PY{l+m+mi}{0}\PY{p}{,} \PY{o}{\PYZhy{}}\PY{n}{lambda2} \PY{p}{,}\PY{l+m+mi}{2} \PY{o}{*} \PY{n}{mu2}\PY{p}{,} \PY{l+m+mi}{0} \PY{p}{,} \PY{l+m+mi}{0} \PY{p}{,} \PY{l+m+mi}{0} \PY{p}{,} \PY{l+m+mi}{0} \PY{p}{,} \PY{l+m+mi}{0} \PY{p}{,} \PY{l+m+mi}{0} \PY{p}{,} \PY{l+m+mi}{0}\PY{p}{,} \PY{l+m+mi}{0}\PY{p}{,} \PY{l+m+mi}{0} \PY{p}{]}
         \PY{n}{e10} \PY{o}{=} \PY{p}{[}\PY{o}{\PYZhy{}}\PY{n}{lambda1}\PY{p}{,} \PY{n}{lambda1} \PY{o}{+} \PY{n}{lambda2} \PY{o}{+} \PY{n}{mu1} \PY{p}{,}\PY{l+m+mi}{0}\PY{p}{,} \PY{o}{\PYZhy{}}\PY{n}{mu2} \PY{p}{,} \PY{l+m+mi}{0} \PY{p}{,} \PY{l+m+mi}{0} \PY{p}{,} \PY{l+m+mi}{0} \PY{p}{,} \PY{o}{\PYZhy{}}\PY{l+m+mi}{2}\PY{o}{*} \PY{n}{mu1} \PY{p}{,} \PY{l+m+mi}{0} \PY{p}{,} \PY{l+m+mi}{0}\PY{p}{,} \PY{l+m+mi}{0}\PY{p}{,} \PY{l+m+mi}{0} \PY{p}{]}
         \PY{n}{e11} \PY{o}{=} \PY{p}{[}\PY{l+m+mi}{0}\PY{p}{,} \PY{o}{\PYZhy{}}\PY{n}{lambda1} \PY{p}{,}\PY{l+m+mi}{0}\PY{p}{,} \PY{n}{lambda1} \PY{o}{+} \PY{n}{mu2} \PY{o}{+} \PY{n}{mu1} \PY{p}{,} \PY{o}{\PYZhy{}}\PY{l+m+mi}{2} \PY{o}{*} \PY{n}{mu1} \PY{p}{,} \PY{l+m+mi}{0} \PY{p}{,} \PY{o}{\PYZhy{}}\PY{n}{lambda2} \PY{p}{,} \PY{l+m+mi}{0} \PY{p}{,} \PY{l+m+mi}{0} \PY{p}{,} \PY{l+m+mi}{0}\PY{p}{,} \PY{l+m+mi}{0}\PY{p}{,} \PY{l+m+mi}{0} \PY{p}{]}
         \PY{n}{e20} \PY{o}{=} \PY{p}{[}\PY{l+m+mi}{0}\PY{p}{,} \PY{l+m+mi}{0} \PY{p}{,}\PY{l+m+mi}{0}\PY{p}{,} \PY{l+m+mi}{0} \PY{p}{,} \PY{o}{\PYZhy{}}\PY{n}{mu2} \PY{p}{,} \PY{l+m+mi}{0} \PY{p}{,} \PY{o}{\PYZhy{}}\PY{n}{lambda1} \PY{p}{,} \PY{n}{lambda1} \PY{o}{+} \PY{n}{lambda2} \PY{o}{+} \PY{l+m+mi}{2} \PY{o}{*} \PY{n}{mu1} \PY{p}{,} \PY{o}{\PYZhy{}}\PY{l+m+mi}{3} \PY{o}{*} \PY{n}{mu1} \PY{p}{,} \PY{l+m+mi}{0}\PY{p}{,} \PY{l+m+mi}{0}\PY{p}{,} \PY{l+m+mi}{0} \PY{p}{]}
         \PY{n}{e21} \PY{o}{=} \PY{p}{[}\PY{l+m+mi}{0}\PY{p}{,} \PY{l+m+mi}{0} \PY{p}{,}\PY{l+m+mi}{0}\PY{p}{,} \PY{o}{\PYZhy{}}\PY{n}{lambda1} \PY{p}{,} \PY{n}{lambda1} \PY{o}{+} \PY{n}{mu2} \PY{o}{+} \PY{l+m+mi}{2} \PY{o}{*} \PY{n}{mu1} \PY{p}{,} \PY{o}{\PYZhy{}}\PY{l+m+mi}{3} \PY{o}{*} \PY{n}{mu1} \PY{p}{,} \PY{l+m+mi}{0} \PY{p}{,} \PY{o}{\PYZhy{}}\PY{n}{lambda2} \PY{p}{,} \PY{l+m+mi}{0} \PY{p}{,} \PY{l+m+mi}{0}\PY{p}{,} \PY{l+m+mi}{0}\PY{p}{,} \PY{l+m+mi}{0} \PY{p}{]}
         \PY{n}{e30} \PY{o}{=} \PY{p}{[}\PY{l+m+mi}{0}\PY{p}{,} \PY{l+m+mi}{0} \PY{p}{,}\PY{l+m+mi}{0}\PY{p}{,} \PY{l+m+mi}{0} \PY{p}{,} \PY{l+m+mi}{0} \PY{p}{,} \PY{o}{\PYZhy{}}\PY{n}{mu2} \PY{p}{,} \PY{l+m+mi}{0} \PY{p}{,} \PY{o}{\PYZhy{}}\PY{n}{lambda1} \PY{p}{,} \PY{n}{lambda1} \PY{o}{+} \PY{l+m+mi}{3} \PY{o}{*} \PY{n}{mu1} \PY{o}{+} \PY{n}{lambda2} \PY{p}{,} \PY{o}{\PYZhy{}}\PY{l+m+mi}{4} \PY{o}{*} \PY{n}{mu1}\PY{p}{,} \PY{l+m+mi}{0}\PY{p}{,} \PY{l+m+mi}{0} \PY{p}{]}
         \PY{n}{e31}\PY{o}{=}  \PY{p}{[}\PY{l+m+mi}{0}\PY{p}{,} \PY{l+m+mi}{0} \PY{p}{,}\PY{l+m+mi}{0}\PY{p}{,} \PY{l+m+mi}{0} \PY{p}{,} \PY{o}{\PYZhy{}}\PY{n}{lambda1} \PY{p}{,} \PY{n}{mu2} \PY{o}{+} \PY{l+m+mi}{3} \PY{o}{*} \PY{n}{mu1} \PY{p}{,} \PY{l+m+mi}{0} \PY{p}{,} \PY{l+m+mi}{0} \PY{p}{,} \PY{o}{\PYZhy{}}\PY{n}{lambda2} \PY{p}{,} \PY{l+m+mi}{0}\PY{p}{,} \PY{l+m+mi}{0}\PY{p}{,} \PY{l+m+mi}{0} \PY{p}{]}
         \PY{n}{e40} \PY{o}{=} \PY{p}{[}\PY{l+m+mi}{0}\PY{p}{,} \PY{l+m+mi}{0} \PY{p}{,}\PY{l+m+mi}{0}\PY{p}{,} \PY{l+m+mi}{0} \PY{p}{,} \PY{l+m+mi}{0} \PY{p}{,}\PY{l+m+mi}{0} \PY{p}{,} \PY{l+m+mi}{0} \PY{p}{,} \PY{l+m+mi}{0} \PY{p}{,} \PY{o}{\PYZhy{}}\PY{n}{lambda1} \PY{p}{,} \PY{n}{lambda1} \PY{o}{+} \PY{l+m+mi}{4} \PY{o}{*} \PY{n}{mu1}\PY{p}{,} \PY{o}{\PYZhy{}}\PY{l+m+mi}{5} \PY{o}{*} \PY{n}{mu1}\PY{p}{,} \PY{l+m+mi}{0} \PY{p}{]}
         \PY{c+c1}{\PYZsh{}e50 = [0, 0 ,0, 0 , 0 ,0 , 0 , 0 , 0 , \PYZhy{}lambda1 , lambda1 + 5 * mu1, \PYZhy{}6 * mu1]}
         
         \PY{c+c1}{\PYZsh{} Sum of all probabilities}
         \PY{n}{e\PYZus{}normalization} \PY{o}{=} \PY{p}{[}\PY{l+m+mi}{1}\PY{p}{,} \PY{l+m+mi}{1} \PY{p}{,}\PY{l+m+mi}{1}\PY{p}{,} \PY{l+m+mi}{1} \PY{p}{,} \PY{l+m+mi}{1} \PY{p}{,}\PY{l+m+mi}{1} \PY{p}{,} \PY{l+m+mi}{1} \PY{p}{,} \PY{l+m+mi}{1} \PY{p}{,} \PY{l+m+mi}{1} \PY{p}{,} \PY{l+m+mi}{1}\PY{p}{,} \PY{l+m+mi}{1}\PY{p}{,} \PY{l+m+mi}{1} \PY{p}{]}
         \PY{n}{e60} \PY{o}{=} \PY{p}{[}\PY{l+m+mi}{0}\PY{p}{,} \PY{l+m+mi}{0} \PY{p}{,}\PY{l+m+mi}{0}\PY{p}{,} \PY{l+m+mi}{0} \PY{p}{,} \PY{l+m+mi}{0} \PY{p}{,}\PY{l+m+mi}{0} \PY{p}{,} \PY{l+m+mi}{0} \PY{p}{,} \PY{l+m+mi}{0} \PY{p}{,} \PY{l+m+mi}{0} \PY{p}{,} \PY{l+m+mi}{0}\PY{p}{,} \PY{o}{\PYZhy{}} \PY{n}{lambda1}\PY{p}{,} \PY{l+m+mi}{6} \PY{o}{*} \PY{n}{mu1} \PY{p}{]}
         
         \PY{n}{matrix\PYZus{}param} \PY{o}{=} \PY{n}{numpy}\PY{o}{.}\PY{n}{matrix}\PY{p}{(}\PY{p}{[}\PY{n}{e00}\PY{p}{,} \PY{n}{e01}\PY{p}{,} \PY{n}{e02}\PY{p}{,} \PY{n}{e10}\PY{p}{,} \PY{n}{e11}\PY{p}{,} \PY{n}{e20}\PY{p}{,} \PY{n}{e21}\PY{p}{,} \PY{n}{e30}\PY{p}{,} \PY{n}{e31}\PY{p}{,} \PY{n}{e40}\PY{p}{,} \PY{n}{e60}\PY{p}{,} \PY{n}{e\PYZus{}normalization}\PY{p}{]}\PY{p}{)}
         \PY{n}{matrix\PYZus{}param\PYZus{}inverse} \PY{o}{=} \PY{n}{numpy}\PY{o}{.}\PY{n}{linalg}\PY{o}{.}\PY{n}{inv}\PY{p}{(}\PY{n}{matrix\PYZus{}param}\PY{p}{)}
         
         
         \PY{n}{matrix\PYZus{}multiple} \PY{o}{=} \PY{n}{numpy}\PY{o}{.}\PY{n}{matrix}\PY{p}{(}\PY{p}{[}\PY{p}{[}\PY{l+m+mi}{0}\PY{p}{]}\PY{p}{,} \PY{p}{[}\PY{l+m+mi}{0}\PY{p}{]}\PY{p}{,} \PY{p}{[}\PY{l+m+mi}{0}\PY{p}{]}\PY{p}{,} \PY{p}{[}\PY{l+m+mi}{0}\PY{p}{]} \PY{p}{,} \PY{p}{[}\PY{l+m+mi}{0}\PY{p}{]}\PY{p}{,} \PY{p}{[}\PY{l+m+mi}{0}\PY{p}{]}\PY{p}{,} \PY{p}{[}\PY{l+m+mi}{0}\PY{p}{]}\PY{p}{,} \PY{p}{[}\PY{l+m+mi}{0}\PY{p}{]} \PY{p}{,} \PY{p}{[}\PY{l+m+mi}{0}\PY{p}{]} \PY{p}{,} \PY{p}{[}\PY{l+m+mi}{0}\PY{p}{]}\PY{p}{,} \PY{p}{[}\PY{l+m+mi}{0}\PY{p}{]}\PY{p}{,} \PY{p}{[}\PY{l+m+mi}{1}\PY{p}{]}\PY{p}{]}\PY{p}{)}
         \PY{n}{matrix\PYZus{}probability} \PY{o}{=} \PY{n}{matrix\PYZus{}param\PYZus{}inverse} \PY{o}{*} \PY{n}{matrix\PYZus{}multiple}
         \PY{c+c1}{\PYZsh{} v00 = (lambda1 + lambda2) * pi(0,0) = mu1 * pi(1,0) +  mu2 * pi(0,1)}
         \PY{c+c1}{\PYZsh{} v01 = (lambda1 + mu2 + lambda2) * pi(0,1) = mu1 * pi(1,1) + (2 * mu2) * pi(0,2) + lambda2 * pi(0, 0)}
         \PY{c+c1}{\PYZsh{} v02 = (2 * mu2) * pi(0,2) = lambda2 * pi(0,1)}
         \PY{c+c1}{\PYZsh{} v10 = (lambda1 + lambda2 + mu1) * pi(1,0) = lambda1 * p(0,0) + mu2 * p(1,1) + 2* mu1 * p(2,0) }
         \PY{c+c1}{\PYZsh{} v11 = (lambda1 + mu2 + mu1) * pi(1,1) = (2 * mu1) * pi(2,1) + lambda1 * pi(0,1) + lambda2 * pi(1,0)}
         \PY{c+c1}{\PYZsh{} v20 = (lambda1 + lambda2 + 2 * mu1) * p(2,0) = lambda1 * pi(1,0) + mu2 * pi(2,1) + 3 * mu1 * p(3,0)}
         \PY{c+c1}{\PYZsh{} v21 = (lambda1 + mu2 + 2 * mu1) * pi(2,1) = (3 * mu1) * pi(3,1) + lambda1 * pi(1,1) + lambda2 * pi(2,0)}
         \PY{c+c1}{\PYZsh{} v30 = (lambda1 + 3 * mu1 + lambda2) * pi(3,0) = lambda1 * pi(2,0) + (4 * mu1) * pi(4,0) +  mu2 * pi(3,1)}
         \PY{c+c1}{\PYZsh{} v31 = (mu2 + 3 * mu1) * pi(3,1) = lambda1 * pi(2,1) + lambda2 * pi(3.0)}
         \PY{c+c1}{\PYZsh{} v40 = (lambda1 + 4 * mu1) * pi(4,0) = lambda1 * pi(3,0) + (5 * mu1) * pi(5,0)}
         \PY{c+c1}{\PYZsh{} v50 = (lambda1 + 5 * mu1) * pi(5,0) = lambda1 * pi(4,0) + (6 * mu1) * pi(6,0)}
         \PY{c+c1}{\PYZsh{} v60 = (6 * mu1) * pi(6,0) = lambda1 * pi(5,0)}
         
         \PY{n}{prob\PYZus{}list} \PY{o}{=} \PY{n}{numpy}\PY{o}{.}\PY{n}{array}\PY{p}{(}\PY{n}{matrix\PYZus{}probability}\PY{p}{)}\PY{o}{.}\PY{n}{reshape}\PY{p}{(}\PY{o}{\PYZhy{}}\PY{l+m+mi}{1}\PY{p}{,}\PY{p}{)}\PY{o}{.}\PY{n}{tolist}\PY{p}{(}\PY{p}{)}
         \PY{n+nb}{print} \PY{p}{(}\PY{l+s+s2}{\PYZdq{}}\PY{l+s+s2}{Equilibrium state probability for each state: }\PY{l+s+se}{\PYZbs{}n}\PY{l+s+s2}{ }\PY{l+s+se}{\PYZbs{}}
         \PY{l+s+s2}{       }\PY{l+s+s2}{\PYZob{}}\PY{l+s+s2}{(0,0), (0,1), (0,2), (1,1), (2,1), (3,1), (1,0), (2,0), (3,0), (4,0), (5,0), (6,0)\PYZcb{} }\PY{l+s+se}{\PYZbs{}n}\PY{l+s+s2}{ }\PY{l+s+s2}{\PYZdq{}}\PY{p}{)}
         
         \PY{n+nb}{print} \PY{p}{(}\PY{n}{matrix\PYZus{}probability}\PY{p}{,} \PY{l+s+s2}{\PYZdq{}}\PY{l+s+se}{\PYZbs{}n}\PY{l+s+s2}{\PYZdq{}}\PY{p}{)}
         \PY{c+c1}{\PYZsh{} 1 only accept }
                         \PY{c+c1}{\PYZsh{}\PYZob{}(0,0), (0,1), (0,2), (1,1), (2,1), (3,1), (1,0), (2,0), (3,0), (4,0), (5,0), (6,0)\PYZcb{} }
                         \PY{c+c1}{\PYZsh{}\PYZob{}(0,0), (0,1), (1,0), (2,0), (3,0)\PYZcb{} }
         \PY{n}{b1} \PY{o}{=} \PY{l+m+mi}{1} \PY{o}{\PYZhy{}} \PY{p}{(}\PY{n}{prob\PYZus{}list}\PY{p}{[}\PY{l+m+mi}{0}\PY{p}{]} \PY{o}{+} \PY{n}{prob\PYZus{}list}\PY{p}{[}\PY{l+m+mi}{1}\PY{p}{]} \PY{o}{+} \PY{n}{prob\PYZus{}list}\PY{p}{[}\PY{l+m+mi}{3}\PY{p}{]} \PY{o}{+} \PY{n}{prob\PYZus{}list}\PY{p}{[}\PY{l+m+mi}{4}\PY{p}{]} \PY{o}{+} \PY{n}{prob\PYZus{}list}\PY{p}{[}\PY{l+m+mi}{6}\PY{p}{]} \PY{o}{+} \PY{n}{prob\PYZus{}list}\PY{p}{[}\PY{l+m+mi}{7}\PY{p}{]} \PY{o}{+} \PY{n}{prob\PYZus{}list}\PY{p}{[}\PY{l+m+mi}{8}\PY{p}{]} \PY{o}{+} \PY{n}{prob\PYZus{}list}\PY{p}{[}\PY{l+m+mi}{9}\PY{p}{]} \PY{o}{+} \PY{n}{prob\PYZus{}list}\PY{p}{[}\PY{l+m+mi}{10}\PY{p}{]}\PY{p}{)}
         \PY{n}{b2} \PY{o}{=} \PY{l+m+mi}{1} \PY{o}{\PYZhy{}} \PY{p}{(}\PY{n}{prob\PYZus{}list}\PY{p}{[}\PY{l+m+mi}{0}\PY{p}{]} \PY{o}{+} \PY{n}{prob\PYZus{}list}\PY{p}{[}\PY{l+m+mi}{1}\PY{p}{]} \PY{o}{+} \PY{n}{prob\PYZus{}list}\PY{p}{[}\PY{l+m+mi}{6}\PY{p}{]} \PY{o}{+} \PY{n}{prob\PYZus{}list}\PY{p}{[}\PY{l+m+mi}{7}\PY{p}{]} \PY{o}{+} \PY{n}{prob\PYZus{}list}\PY{p}{[}\PY{l+m+mi}{8}\PY{p}{]}\PY{p}{)}
         
         \PY{n+nb}{print} \PY{p}{(}\PY{n}{b1}\PY{p}{)}
         \PY{n+nb}{print} \PY{p}{(}\PY{n}{b2}\PY{p}{)}
\end{Verbatim}


    \begin{Verbatim}[commandchars=\\\{\}]
Equilibrium state probability for each state: 
        \{(0,0), (0,1), (0,2), (1,1), (2,1), (3,1), (1,0), (2,0), (3,0), (4,0), (5,0), (6,0)\} 
 
[[-1.65059210e-04]
 [ 1.45875334e-06]
 [ 4.55860421e-08]
 [ 2.30587955e-05]
 [ 1.82422573e-04]
 [ 9.21913715e-04]
 [-2.47677389e-03]
 [ 1.24846572e-03]
 [ 1.94550288e-02]
 [ 8.28732337e-02]
 [ 2.56553202e-01]
 [ 6.41383004e-01]] 

0.6423049632545225
0.9819368798599208

    \end{Verbatim}

    

    \subsection{Use the product-form theorem discussed during the lecture of
March 5 to give an explicit formula for the equilibrium state
probabilities of the Markov
chain.}\label{use-the-product-form-theorem-discussed-during-the-lecture-of-march-5-to-give-an-explicit-formula-for-the-equilibrium-state-probabilities-of-the-markov-chain.}

    \$K = 2 \$ call classes

load \$ ro\_k = lambda\_j * d\_j \$ (k = 1,...,K)

\(n_k\) := number of class-k calls in progress (k = 1,...,K)

\(\pi_n = 1/G \prod_{j=1}^{K} \frac{\rho_{j}^nj}{n_{j}}\)

Where:

\$G := \sum\emph{\{N\in S\}\prod}\{j=1\}\^{}\{K\}
\frac{\rho_{j}^nj}{n_{j}} \$

With:

\$ \rho\_\{j\} := \lambda\_j\delta\_j\$

So G is the sum of all possible states for each class. When we divide
this to 1, we have a normalizing constant. The we can multiply this by
the product of each state in a certain class, and we have the blocking
probability.

    

    \subsection{Calculate the equilibrium state probabilities for the model
(simply by filling out the proper parameter value in the formula
obtained in question
8).}\label{calculate-the-equilibrium-state-probabilities-for-the-model-simply-by-filling-out-the-proper-parameter-value-in-the-formula-obtained-in-question-8.}

    \begin{Verbatim}[commandchars=\\\{\}]
{\color{incolor}In [{\color{incolor}27}]:} \PY{k+kn}{import} \PY{n+nn}{math}
         \PY{k+kn}{import} \PY{n+nn}{numpy}
         \PY{n}{K} \PY{o}{=} \PY{l+m+mi}{2}
         \PY{n}{states} \PY{o}{=} \PY{p}{[}\PY{p}{(}\PY{l+m+mi}{0}\PY{p}{,}\PY{l+m+mi}{0}\PY{p}{)}\PY{p}{,} \PY{p}{(}\PY{l+m+mi}{0}\PY{p}{,}\PY{l+m+mi}{1}\PY{p}{)}\PY{p}{,} \PY{p}{(}\PY{l+m+mi}{0}\PY{p}{,}\PY{l+m+mi}{2}\PY{p}{)}\PY{p}{,} \PY{p}{(}\PY{l+m+mi}{1}\PY{p}{,}\PY{l+m+mi}{1}\PY{p}{)}\PY{p}{,} \PY{p}{(}\PY{l+m+mi}{2}\PY{p}{,}\PY{l+m+mi}{1}\PY{p}{)}\PY{p}{,} \PY{p}{(}\PY{l+m+mi}{3}\PY{p}{,}\PY{l+m+mi}{1}\PY{p}{)}\PY{p}{,} \PY{p}{(}\PY{l+m+mi}{1}\PY{p}{,}\PY{l+m+mi}{0}\PY{p}{)}\PY{p}{,} \PY{p}{(}\PY{l+m+mi}{2}\PY{p}{,}\PY{l+m+mi}{0}\PY{p}{)}\PY{p}{,} \PY{p}{(}\PY{l+m+mi}{3}\PY{p}{,}\PY{l+m+mi}{0}\PY{p}{)}\PY{p}{,} \PY{p}{(}\PY{l+m+mi}{4}\PY{p}{,}\PY{l+m+mi}{0}\PY{p}{)}\PY{p}{,} \PY{p}{(}\PY{l+m+mi}{5}\PY{p}{,}\PY{l+m+mi}{0}\PY{p}{)}\PY{p}{,} \PY{p}{(}\PY{l+m+mi}{6}\PY{p}{,}\PY{l+m+mi}{0}\PY{p}{)}\PY{p}{]}
         \PY{n}{lamda} \PY{o}{=} \PY{p}{[}\PY{l+m+mi}{10}\PY{p}{,} \PY{l+m+mf}{0.25}\PY{o}{/}\PY{l+m+mi}{60}\PY{p}{]}
         \PY{n}{d} \PY{o}{=} \PY{p}{[}\PY{l+m+mi}{1}\PY{o}{/}\PY{l+m+mf}{1.5}\PY{p}{,} \PY{l+m+mi}{1}\PY{o}{/}\PY{l+m+mi}{15}\PY{p}{]}
         
         \PY{n}{G\PYZus{}calc} \PY{o}{=} \PY{p}{[}\PY{p}{]}
         \PY{n}{pf} \PY{o}{=} \PY{p}{[}\PY{p}{[}\PY{p}{]}\PY{p}{,}\PY{p}{[}\PY{p}{]}\PY{p}{]}
         \PY{n}{equilibrium} \PY{o}{=} \PY{p}{[}\PY{p}{]}
         
         \PY{k}{for} \PY{n}{state} \PY{o+ow}{in} \PY{n}{states}\PY{p}{:}
             \PY{k}{for} \PY{n}{j} \PY{o+ow}{in} \PY{n+nb}{range}\PY{p}{(}\PY{n}{K}\PY{p}{)}\PY{p}{:}
                 \PY{n}{ro} \PY{o}{=} \PY{n}{lamda}\PY{p}{[}\PY{n}{j}\PY{p}{]} \PY{o}{*} \PY{n}{d}\PY{p}{[}\PY{n}{j}\PY{p}{]}
                 \PY{n}{G\PYZus{}calc}\PY{o}{.}\PY{n}{append}\PY{p}{(}\PY{n}{ro} \PY{o}{*}\PY{o}{*} \PY{n}{state}\PY{p}{[}\PY{n}{j}\PY{p}{]} \PY{o}{/} \PY{n}{math}\PY{o}{.}\PY{n}{factorial}\PY{p}{(}\PY{n}{state}\PY{p}{[}\PY{n}{j}\PY{p}{]}\PY{p}{)}\PY{p}{)}
         
         \PY{n}{G} \PY{o}{=} \PY{n}{numpy}\PY{o}{.}\PY{n}{product}\PY{p}{(}\PY{n}{G\PYZus{}calc}\PY{p}{)}
         
         \PY{k}{for} \PY{n}{state} \PY{o+ow}{in} \PY{n}{states}\PY{p}{:}
             \PY{k}{for} \PY{n}{j} \PY{o+ow}{in} \PY{n+nb}{range}\PY{p}{(}\PY{n}{K}\PY{p}{)}\PY{p}{:}
                 \PY{n}{ro} \PY{o}{=} \PY{n}{lamda}\PY{p}{[}\PY{n}{j}\PY{p}{]} \PY{o}{*} \PY{n}{d}\PY{p}{[}\PY{n}{j}\PY{p}{]}
                 \PY{n}{pf}\PY{p}{[}\PY{n}{j}\PY{p}{]}\PY{o}{.}\PY{n}{append}\PY{p}{(}\PY{n}{ro} \PY{o}{*}\PY{o}{*} \PY{n}{state}\PY{p}{[}\PY{n}{j}\PY{p}{]} \PY{o}{/} \PY{n}{math}\PY{o}{.}\PY{n}{factorial}\PY{p}{(}\PY{n}{state}\PY{p}{[}\PY{n}{j}\PY{p}{]}\PY{p}{)}\PY{p}{)}
         
         \PY{k}{for} \PY{n}{p} \PY{o+ow}{in} \PY{n}{pf}\PY{p}{:}
             \PY{n}{equilibrium}\PY{o}{.}\PY{n}{append}\PY{p}{(}\PY{p}{(}\PY{l+m+mi}{1}\PY{o}{/}\PY{n}{G}\PY{p}{)} \PY{o}{*} \PY{n}{numpy}\PY{o}{.}\PY{n}{product}\PY{p}{(}\PY{n}{p}\PY{p}{)}\PY{p}{)}
         
         \PY{n+nb}{print}\PY{p}{(}\PY{n}{equilibrium}\PY{p}{,} \PY{n}{numpy}\PY{o}{.}\PY{n}{sum}\PY{p}{(}\PY{n}{equilibrium}\PY{p}{)}\PY{p}{)}
\end{Verbatim}


    \begin{Verbatim}[commandchars=\\\{\}]
[4.3535646720000016e+21, 1.6964906514146546e-14] 4.3535646720000016e+21

    \end{Verbatim}

    \subsection{Use this result to calculate the blocking probabilities for
both call
classes.}\label{use-this-result-to-calculate-the-blocking-probabilities-for-both-call-classes.}

    \subsection{Formulate the Kaufman-Roberts recursion for the
model.}\label{formulate-the-kaufman-roberts-recursion-for-the-model.}

     

    \begin{Verbatim}[commandchars=\\\{\}]
{\color{incolor}In [{\color{incolor}28}]:} \PY{k}{def} \PY{n+nf}{getBlockProbabilityForKClass}\PY{p}{(}\PY{p}{)}\PY{p}{:}
           \PY{n}{b} \PY{o}{=} \PY{p}{(}\PY{l+m+mi}{1}\PY{p}{,} \PY{l+m+mi}{3}\PY{p}{)} \PY{c+c1}{\PYZsh{} Tuple of required channels for type of connection}
           \PY{n}{d} \PY{o}{=} \PY{p}{(}\PY{l+m+mf}{1.5}\PY{p}{,} \PY{l+m+mi}{15}\PY{p}{)} \PY{c+c1}{\PYZsh{} Tuple of call duration}
           
           \PY{n}{lamda} \PY{o}{=} \PY{p}{(}\PY{l+m+mi}{10}\PY{p}{,} \PY{l+m+mf}{0.25}\PY{o}{/}\PY{l+m+mi}{60}\PY{p}{)} \PY{c+c1}{\PYZsh{} arrival rate}
           \PY{n}{p} \PY{o}{=} \PY{p}{(}\PY{n}{lamda}\PY{p}{[}\PY{l+m+mi}{0}\PY{p}{]} \PY{o}{*} \PY{n}{d}\PY{p}{[}\PY{l+m+mi}{0}\PY{p}{]}\PY{p}{,} \PY{n}{lamda}\PY{p}{[}\PY{l+m+mi}{1}\PY{p}{]} \PY{o}{*} \PY{n}{d}\PY{p}{[}\PY{l+m+mi}{1}\PY{p}{]}\PY{p}{)} \PY{c+c1}{\PYZsh{} load}
         
           \PY{n}{C} \PY{o}{=} \PY{l+m+mi}{6} \PY{c+c1}{\PYZsh{} Number of channel}
           \PY{n}{K} \PY{o}{=} \PY{l+m+mi}{2} \PY{c+c1}{\PYZsh{} Number of class}
         
           \PY{n}{g} \PY{o}{=} \PY{p}{[}\PY{k+kc}{None}\PY{p}{]} \PY{o}{*} \PY{p}{(}\PY{n}{C}\PY{o}{+}\PY{l+m+mi}{1}\PY{p}{)}
           \PY{n}{g}\PY{p}{[}\PY{l+m+mi}{0}\PY{p}{]} \PY{o}{=} \PY{l+m+mi}{1}
         
           \PY{c+c1}{\PYZsh{} Calculate g array}
           \PY{k}{for} \PY{n}{c} \PY{o+ow}{in} \PY{n+nb}{range}\PY{p}{(}\PY{l+m+mi}{1}\PY{p}{,} \PY{n}{C}\PY{o}{+}\PY{l+m+mi}{1}\PY{p}{)}\PY{p}{:}
             \PY{n}{gc\PYZus{}temp} \PY{o}{=} \PY{l+m+mi}{0}
             \PY{k}{for} \PY{n}{j} \PY{o+ow}{in} \PY{n+nb}{range}\PY{p}{(}\PY{n}{K}\PY{p}{)}\PY{p}{:}
               \PY{n}{g\PYZus{}cb\PYZus{}index} \PY{o}{=} \PY{n}{c}\PY{o}{\PYZhy{}}\PY{n}{b}\PY{p}{[}\PY{n}{j}\PY{p}{]}
               \PY{k}{if} \PY{p}{(}\PY{n}{g\PYZus{}cb\PYZus{}index} \PY{o}{\PYZgt{}}\PY{o}{=}\PY{l+m+mi}{0}\PY{p}{)}\PY{p}{:}
                 \PY{n}{gc\PYZus{}temp} \PY{o}{+}\PY{o}{=} \PY{n}{b}\PY{p}{[}\PY{n}{j}\PY{p}{]} \PY{o}{*} \PY{n}{p}\PY{p}{[}\PY{n}{j}\PY{p}{]} \PY{o}{*} \PY{n}{g}\PY{p}{[}\PY{n}{g\PYZus{}cb\PYZus{}index}\PY{p}{]}
               
             \PY{n}{gc\PYZus{}temp} \PY{o}{=} \PY{n}{gc\PYZus{}temp} \PY{o}{*} \PY{l+m+mi}{1}\PY{o}{/}\PY{n}{c}
             \PY{n}{g}\PY{p}{[}\PY{n}{c}\PY{p}{]} \PY{o}{=} \PY{n}{gc\PYZus{}temp}
             
             
           \PY{c+c1}{\PYZsh{} Calculate G\PYZus{}Constant}
           \PY{n}{G} \PY{o}{=} \PY{n+nb}{sum}\PY{p}{(}\PY{n}{g}\PY{p}{)}
          
         
           \PY{c+c1}{\PYZsh{} Equilibrium distribution for the number of occupied resources}
           \PY{n}{q} \PY{o}{=} \PY{p}{[}\PY{k+kc}{None}\PY{p}{]} \PY{o}{*} \PY{n+nb}{len}\PY{p}{(}\PY{n}{g}\PY{p}{)}
           \PY{k}{for} \PY{n}{c} \PY{o+ow}{in} \PY{n+nb}{range}\PY{p}{(}\PY{n}{C}\PY{o}{+}\PY{l+m+mi}{1}\PY{p}{)}\PY{p}{:}
             \PY{n}{q}\PY{p}{[}\PY{n}{c}\PY{p}{]} \PY{o}{=} \PY{n}{g}\PY{p}{[}\PY{n}{c}\PY{p}{]}\PY{o}{/}\PY{n}{G}
         
           \PY{c+c1}{\PYZsh{} Blocking probability result}
           \PY{n}{B} \PY{o}{=} \PY{p}{[}\PY{k+kc}{None}\PY{p}{]} \PY{o}{*} \PY{n}{K}
           \PY{k}{for} \PY{n}{j} \PY{o+ow}{in} \PY{n+nb}{range}\PY{p}{(}\PY{n}{K}\PY{p}{)}\PY{p}{:}
             \PY{n}{B\PYZus{}temp} \PY{o}{=} \PY{l+m+mi}{0}
             \PY{k}{for} \PY{n}{c} \PY{o+ow}{in} \PY{n+nb}{range}\PY{p}{(}\PY{n}{C}\PY{o}{\PYZhy{}}\PY{n}{b}\PY{p}{[}\PY{n}{j}\PY{p}{]}\PY{o}{+}\PY{l+m+mi}{1}\PY{p}{,} \PY{n}{C}\PY{o}{+}\PY{l+m+mi}{1}\PY{p}{)}\PY{p}{:}
               \PY{n}{B\PYZus{}temp} \PY{o}{+}\PY{o}{=} \PY{n}{q}\PY{p}{[}\PY{n}{c}\PY{p}{]}
             \PY{n}{B}\PY{p}{[}\PY{n}{j}\PY{p}{]} \PY{o}{=} \PY{n+nb}{format}\PY{p}{(}\PY{n}{B\PYZus{}temp}\PY{p}{,} \PY{l+s+s1}{\PYZsq{}}\PY{l+s+s1}{.3f}\PY{l+s+s1}{\PYZsq{}}\PY{p}{)}
         
           \PY{k}{return} \PY{n}{B}
         
         \PY{n}{blockProbabilityArr} \PY{o}{=} \PY{n}{getBlockProbabilityForKClass}\PY{p}{(}\PY{p}{)}
         
         \PY{n+nb}{print}\PY{p}{(}\PY{l+s+s1}{\PYZsq{}}\PY{l+s+s1}{====== Block probability for class=======}\PY{l+s+s1}{\PYZsq{}}\PY{p}{)}
         \PY{k}{for} \PY{n}{index} \PY{o+ow}{in} \PY{n+nb}{range}\PY{p}{(}\PY{n+nb}{len}\PY{p}{(}\PY{n}{blockProbabilityArr}\PY{p}{)}\PY{p}{)}\PY{p}{:}
           \PY{n+nb}{print} \PY{p}{(}\PY{l+s+s1}{\PYZsq{}}\PY{l+s+s1}{Class }\PY{l+s+s1}{\PYZsq{}}\PY{p}{,} \PY{n}{index}\PY{p}{,} \PY{l+s+s1}{\PYZsq{}}\PY{l+s+s1}{: }\PY{l+s+s1}{\PYZsq{}}\PY{p}{,} \PY{n}{blockProbabilityArr}\PY{p}{[}\PY{n}{index}\PY{p}{]}\PY{p}{)}
\end{Verbatim}


    \begin{Verbatim}[commandchars=\\\{\}]
====== Block probability for class=======
Class  0 :  0.634
Class  1 :  0.972

    \end{Verbatim}

    \section{Three cell analysis}\label{three-cell-analysis}

    Consider a cellular voice network with three neighboring cells, as
depicted below. The mean number of call attempts per minute for cells 1,
2 and 3 are 5, 10 and 15, respectively. We assume that the mean call
duration is 2 minutes. There are a total of 32 channels, and there is a
fixed number of channels per cell. However, to avoid interference
neighboring cells cannot use the same channel.

    \subsection{Determine the distribution of the 32 channels over the three
cells that minimizes the call blocking probability of an arbitrary call
(regardless of the cell in which takes place. To this end, extend your
Erlang-B calculator to the three-cell case, and sure it calculates the
call blocking probabilities per cell and the overall call blocking
probability.}\label{determine-the-distribution-of-the-32-channels-over-the-three-cells-that-minimizes-the-call-blocking-probability-of-an-arbitrary-call-regardless-of-the-cell-in-which-takes-place.-to-this-end-extend-your-erlang-b-calculator-to-the-three-cell-case-and-sure-it-calculates-the-call-blocking-probabilities-per-cell-and-the-overall-call-blocking-probability.}

    \subsection{Suppose we want to have an overall call blocking probability
less than 5\%. Do we need any additional channels? If so, how many
additional channels are needed, and what would the optimal allocation of
these channels then
be?}\label{suppose-we-want-to-have-an-overall-call-blocking-probability-less-than-5.-do-we-need-any-additional-channels-if-so-how-many-additional-channels-are-needed-and-what-would-the-optimal-allocation-of-these-channels-then-be}

    \begin{Verbatim}[commandchars=\\\{\}]
{\color{incolor}In [{\color{incolor}29}]:} \PY{k+kn}{import} \PY{n+nn}{math}
         
         \PY{k}{def} \PY{n+nf}{try\PYZus{}backtrack}\PY{p}{(}\PY{n}{i}\PY{p}{,} \PY{n}{triple}\PY{p}{,} \PY{n}{result\PYZus{}subsets}\PY{p}{,} \PY{n}{space}\PY{p}{,} \PY{n}{total}\PY{p}{)}\PY{p}{:}
             \PY{k}{for} \PY{n}{index} \PY{o+ow}{in} \PY{n+nb}{range}\PY{p}{(}\PY{n+nb}{len}\PY{p}{(}\PY{n}{space}\PY{p}{)}\PY{p}{)}\PY{p}{:}
                 \PY{n}{triple}\PY{p}{[}\PY{n}{i}\PY{p}{]} \PY{o}{=} \PY{n}{space}\PY{p}{[}\PY{n}{index}\PY{p}{]}
                 
                 \PY{k}{if} \PY{p}{(}\PY{n}{i} \PY{o}{==} \PY{p}{(}\PY{n+nb}{len}\PY{p}{(}\PY{n}{triple}\PY{p}{)}\PY{o}{\PYZhy{}}\PY{l+m+mi}{1}\PY{p}{)} \PY{o+ow}{and} \PY{n+nb}{sum}\PY{p}{(}\PY{n}{triple}\PY{p}{)} \PY{o}{==} \PY{n}{total}\PY{p}{)}\PY{p}{:}
                     \PY{n}{copy\PYZus{}tripe} \PY{o}{=} \PY{n+nb}{list}\PY{p}{(}\PY{n}{triple}\PY{p}{)}
                     \PY{k}{if} \PY{n}{copy\PYZus{}tripe} \PY{o}{!=} \PY{p}{[}\PY{p}{]}\PY{p}{:}
                         \PY{n}{result\PYZus{}subsets}\PY{o}{.}\PY{n}{append}\PY{p}{(}\PY{n}{copy\PYZus{}tripe}\PY{p}{)}
         
                 \PY{k}{if} \PY{p}{(}\PY{n}{i} \PY{o}{!=} \PY{n+nb}{len}\PY{p}{(}\PY{n}{triple}\PY{p}{)} \PY{o}{\PYZhy{}} \PY{l+m+mi}{1}\PY{p}{)}\PY{p}{:}    
                     \PY{n}{try\PYZus{}backtrack}\PY{p}{(}\PY{n}{i} \PY{o}{+} \PY{l+m+mi}{1}\PY{p}{,} \PY{n}{triple}\PY{p}{,} \PY{n}{result\PYZus{}subsets}\PY{p}{,} \PY{n}{space}\PY{p}{,} \PY{n}{total}\PY{p}{)}
         
         
         \PY{k}{def} \PY{n+nf}{generateAllSubsetSumToN}\PY{p}{(}\PY{n}{space}\PY{p}{,} \PY{n}{n} \PY{p}{,} \PY{n}{total}\PY{p}{)}\PY{p}{:}
             \PY{n}{subsets} \PY{o}{=} \PY{p}{[}\PY{p}{[}\PY{p}{]}\PY{p}{]}
             \PY{n}{backtrack\PYZus{}triple} \PY{o}{=} \PY{p}{[}\PY{k+kc}{None}\PY{p}{]} \PY{o}{*} \PY{n}{n}
             \PY{n}{try\PYZus{}backtrack}\PY{p}{(}\PY{l+m+mi}{0}\PY{p}{,} \PY{n}{backtrack\PYZus{}triple}\PY{p}{,} \PY{n}{subsets}\PY{p}{,} \PY{n}{space}\PY{p}{,} \PY{n}{total}\PY{p}{)}
             \PY{k}{return} \PY{n}{subsets}
         
         \PY{c+c1}{\PYZsh{}\PYZsh{}\PYZsh{}\PYZsh{}\PYZsh{}\PYZsh{}\PYZsh{} Erlang model}
         \PY{n}{lamda1} \PY{o}{=} \PY{l+m+mi}{5}
         \PY{n}{lamda2} \PY{o}{=} \PY{l+m+mi}{10}
         \PY{n}{lamda3} \PY{o}{=} \PY{l+m+mi}{15}
         \PY{n}{lamda\PYZus{}sum} \PY{o}{=} \PY{n}{lamda1} \PY{o}{+} \PY{n}{lamda2} \PY{o}{+} \PY{n}{lamda3}
         
         \PY{n}{beta} \PY{o}{=} \PY{l+m+mi}{2}
         \PY{n}{lamda1\PYZus{}average} \PY{o}{=} \PY{p}{(}\PY{n}{lamda1}\PY{p}{)}\PY{o}{/}\PY{n}{lamda\PYZus{}sum}
         \PY{n}{lamda2\PYZus{}average} \PY{o}{=} \PY{p}{(}\PY{n}{lamda2}\PY{p}{)}\PY{o}{/}\PY{n}{lamda\PYZus{}sum}
         \PY{n}{lamda3\PYZus{}average} \PY{o}{=} \PY{p}{(}\PY{n}{lamda3}\PY{p}{)}\PY{o}{/}\PY{n}{lamda\PYZus{}sum}
         
         \PY{c+c1}{\PYZsh{} Calculate upper part of Erlang\PYZhy{}B formula}
         \PY{k}{def} \PY{n+nf}{bp\PYZus{}calc}\PY{p}{(}\PY{n}{lamda}\PY{p}{,} \PY{n}{beta}\PY{p}{,} \PY{n}{N}\PY{p}{)}\PY{p}{:}
             \PY{n}{bp} \PY{o}{=} \PY{p}{(}\PY{n}{lamda} \PY{o}{*} \PY{n}{beta}\PY{p}{)} \PY{o}{*}\PY{o}{*} \PY{n}{N} \PY{o}{/} \PY{n}{math}\PY{o}{.}\PY{n}{factorial}\PY{p}{(}\PY{n}{N}\PY{p}{)}
             \PY{k}{return} \PY{n}{bp}
         
         \PY{c+c1}{\PYZsh{} Calculate Markov chain}
         \PY{k}{def} \PY{n+nf}{erlang\PYZus{}B}\PY{p}{(}\PY{n}{lamda}\PY{p}{,} \PY{n}{beta}\PY{p}{,} \PY{n}{N}\PY{p}{)}\PY{p}{:}
             \PY{n}{bp1} \PY{o}{=} \PY{n}{bp\PYZus{}calc}\PY{p}{(}\PY{n}{lamda}\PY{p}{,} \PY{n}{beta}\PY{p}{,} \PY{n}{N}\PY{p}{)}
             \PY{n}{bp2} \PY{o}{=} \PY{l+m+mi}{1}
             
             \PY{k}{for} \PY{n}{i} \PY{o+ow}{in} \PY{n+nb}{range}\PY{p}{(}\PY{n}{N}\PY{p}{)}\PY{p}{:}
                 \PY{n}{bp2} \PY{o}{+}\PY{o}{=} \PY{n}{bp\PYZus{}calc}\PY{p}{(}\PY{n}{lamda}\PY{p}{,} \PY{n}{beta}\PY{p}{,} \PY{n}{i} \PY{o}{+} \PY{l+m+mi}{1}\PY{p}{)}
                 
             \PY{k}{return} \PY{n}{bp1} \PY{o}{/} \PY{n}{bp2}
         
         \PY{c+c1}{\PYZsh{}\PYZsh{}\PYZsh{}\PYZsh{}\PYZsh{}\PYZsh{}\PYZsh{}\PYZsh{} Collect all probability of arbitrary in 3 cells}
         \PY{n}{space} \PY{o}{=} \PY{n+nb}{list}\PY{p}{(}\PY{n+nb}{range}\PY{p}{(}\PY{l+m+mi}{1}\PY{p}{,} \PY{l+m+mi}{33}\PY{p}{)}\PY{p}{)}
         \PY{n}{subsets} \PY{o}{=} \PY{n}{generateAllSubsetSumToN}\PY{p}{(}\PY{n}{space}\PY{p}{,} \PY{l+m+mi}{3}\PY{p}{,} \PY{l+m+mi}{32}\PY{p}{)}
         \PY{n}{subsets} \PY{o}{=} \PY{p}{[}\PY{n}{x} \PY{k}{for} \PY{n}{x} \PY{o+ow}{in} \PY{n}{subsets} \PY{k}{if} \PY{n}{x} \PY{o}{!=} \PY{p}{[}\PY{p}{]}\PY{p}{]}
         
         \PY{n}{nsubset} \PY{o}{=} \PY{n+nb}{len}\PY{p}{(}\PY{n}{subsets}\PY{p}{)}
         \PY{n}{prob} \PY{o}{=} \PY{p}{[}\PY{k+kc}{None}\PY{p}{]} \PY{o}{*} \PY{n}{nsubset}
         \PY{k}{for} \PY{n}{index} \PY{o+ow}{in} \PY{n+nb}{range}\PY{p}{(}\PY{n}{nsubset}\PY{p}{)}\PY{p}{:}
             \PY{n}{subset} \PY{o}{=} \PY{n}{subsets}\PY{p}{[}\PY{n}{index}\PY{p}{]}
             \PY{n}{prob}\PY{p}{[}\PY{n}{index}\PY{p}{]} \PY{o}{=} \PY{n}{lamda1\PYZus{}average} \PY{o}{*} \PY{n}{erlang\PYZus{}B}\PY{p}{(}\PY{n}{lamda1}\PY{p}{,} \PY{n}{beta}\PY{p}{,} \PY{n}{subset}\PY{p}{[}\PY{l+m+mi}{0}\PY{p}{]}\PY{p}{)} \PY{o}{+} \PY{n}{lamda2\PYZus{}average} \PY{o}{*} \PY{n}{erlang\PYZus{}B}\PY{p}{(}\PY{n}{lamda2}\PY{p}{,} \PY{n}{beta}\PY{p}{,} \PY{n}{subset}\PY{p}{[}\PY{l+m+mi}{1}\PY{p}{]}\PY{p}{)} \PY{o}{+} \PY{n}{lamda3\PYZus{}average} \PY{o}{*} \PY{n}{erlang\PYZus{}B}\PY{p}{(}\PY{n}{lamda3}\PY{p}{,} \PY{n}{beta}\PY{p}{,} \PY{n}{subset}\PY{p}{[}\PY{l+m+mi}{2}\PY{p}{]}\PY{p}{)}
             
         \PY{n}{min\PYZus{}prob} \PY{o}{=} \PY{n+nb}{min}\PY{p}{(}\PY{n}{prob}\PY{p}{)}
         \PY{n+nb}{print}\PY{p}{(}\PY{l+s+s2}{\PYZdq{}}\PY{l+s+s2}{Min probability of blocking: }\PY{l+s+s2}{\PYZdq{}}\PY{p}{,} \PY{n}{min\PYZus{}prob}\PY{p}{)}
         \PY{n}{set\PYZus{}index\PYZus{}for\PYZus{}min\PYZus{}prob} \PY{o}{=} \PY{n}{prob}\PY{o}{.}\PY{n}{index}\PY{p}{(}\PY{n}{min\PYZus{}prob}\PY{p}{)}
         \PY{n+nb}{print}\PY{p}{(}\PY{l+s+s2}{\PYZdq{}}\PY{l+s+s2}{Number of channels to minimize blocking prob of three cells: }\PY{l+s+s2}{\PYZdq{}}\PY{p}{,} \PY{n}{subsets}\PY{p}{[}\PY{n}{set\PYZus{}index\PYZus{}for\PYZus{}min\PYZus{}prob}\PY{p}{]}\PY{p}{)}
\end{Verbatim}


    \begin{Verbatim}[commandchars=\\\{\}]
Min probability of blocking:  0.5057064834151516
Number of channels to minimize blocking prob of three cells:  [3, 10, 19]

    \end{Verbatim}


    % Add a bibliography block to the postdoc
    
    
    
    \end{document}
